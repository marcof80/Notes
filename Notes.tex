%Prefacio
%Documento
\documentclass[10.5pt, a4paper]{article}

% Paquetes
\usepackage[margin=1in]{geometry}
\usepackage{graphicx}              
\usepackage{amsmath}               
\usepackage{amsfonts}              
\usepackage{amsthm}
\usepackage{enumerate}
\usepackage{tikz-cd}
\usepackage[colorlinks=true , linktoc=page]{hyperref}
\usepackage{url}        
\usepackage[full]{complexity}        %que ordenado maquina jaja
 
%\usepackage[spanish]{babel}
%\usepackage[utf8]{inputenc}

%Tablas
\usepackage{float}
\restylefloat{table}

% Layout
\usepackage{fancyhdr}
\setlength{\headheight}{15.2pt}
\pagestyle{fancy}

% Teoremas, etc
% \newtheorem{thm}{Teorema}[section]
% \newtheorem{lem}[thm]{Lema}
% \newtheorem{prop}[thm]{Proposición}
% \newtheorem*{prop*}{Proposición}
% \newtheorem{cor}[thm]{Corolario}
% \newtheorem{conj}[thm]{Conjetura}
% \newtheorem{ej}[thm]{Ejemplo}
% \newtheorem{ax}[thm]{Axioma}

%en ingles..
 \newtheorem{thm}{Theorem}[section]
 \newtheorem{lem}[thm]{Lemma}
 \newtheorem{prop}[thm]{Proposition}
 \newtheorem*{prop*}{Proposition}
 \newtheorem{cor}[thm]{Corollary}
 \newtheorem{conj}[thm]{Claim}
 \newtheorem{ej}[thm]{Example}
 \newtheorem{ax}[thm]{Axiom}


% % Renombrar Comandos
% \newcommand{\bd}[1]{\mathbf{#1}}
% \newcommand{\R}{\mathbb{R}}      
% \newcommand{\Z}{\mathbb{Z}}     
% \newcommand{\N}{\mathbb{N}}
% \newcommand{\Ra}{\rightarrow}
% \newcommand{\U}{\mathcal{U}}
% \newcommand{\col}[1]{\left[\begin{matrix} #1 \end{matrix} \right]}
% \newcommand{\comb}[2]{\binom{#1^2 + #2^2}{#1+#2}}

% %Highlight text
% \usepackage{color}
% \newcommand{\hilight}[1]{\colorbox{yellow}{#1}}


\title{Category Theory and \\ Computational Complexity}
\author{Marco Larrea \and Octavio Zapata}

\begin{document}
\maketitle
A first-order dependence logic $D$ is a class which consists of all $D$-definable properties where $D := (FO + \mu.\bar{t})$ and $\mu.\bar{t}$ denotes that term $t_{|\bar{t}|}$ is functionally dependent on $t_{i}$ for all $i\leq |\bar{t}|$. The model class $FO$ is as always defined as the class of models of all first-order sentences (i.e. $FO:= \{S:(\exists\tau)(\exists\varphi\in L(\tau))\ S=Mod(\varphi)\}$ where $L(\tau)$ is a first-order language of type $\tau$) and $\mu.\bar{t}$ is interpreted as a recursively generated tuple of terms which we naturally identify with the set $[|\bar{t}|] := \{1,2,\dots,|\bar{t}|\}$. $D$ sentences are capable to characterise variable dependence and in general they are proven to be as expressive as the sentences of the second order $\Sigma_1^1$ fragment. The intuitionistic dependence version $ID$ has the same expressive power as full $SO$. It is a fact that $MID$-model checking is $PSPACE$-complete where $MID$ is the intuitionistic implication fragment of the modal dependence logic $MD$ which contains at least two modifiers. Hence, $(FO + \mu.\bar{t}) = NP, ID = \Sigma_{\ast}P$ and $MID = PSPACE$.  On the other hand, $PSPACE = IP = QIP$ and so $MID = QIP$ the quantum version of the interactive polytime class.

We shall try to cook up a purely algebraic definition for the class of structures $MID$ and extend such categorical logic in order to capture other quantum and classical complexity classes. 

%Referencias
\nocite{*}
\bibliographystyle{alpha}
\bibliography{paper}

\end{document}
