%Documento
\documentclass[10.5pt, a4paper]{article}

% Paquetes
\usepackage[margin=1in]{geometry}
\usepackage{graphicx}              
\usepackage{amsmath}               
\usepackage{amsfonts}              
\usepackage{amsthm}
\usepackage{enumerate}
\usepackage{tikz-cd}
\usepackage[colorlinks=true , linktoc=page]{hyperref}
\usepackage{url}        
\usepackage[full]{complexity}        %que ordenado maquina jaja
 
%\usepackage[spanish]{babel}
%\usepackage[utf8]{inputenc}

%Tablas
\usepackage{float}
\restylefloat{table}

% Layout
\usepackage{fancyhdr}
\setlength{\headheight}{15.2pt}
\pagestyle{fancy}

% Teoremas, etc
% \newtheorem{thm}{Teorema}[section]
% \newtheorem{lem}[thm]{Lema}
% \newtheorem{prop}[thm]{Proposición}
% \newtheorem*{prop*}{Proposición}
% \newtheorem{cor}[thm]{Corolario}
% \newtheorem{conj}[thm]{Conjetura}
% \newtheorem{ej}[thm]{Ejemplo}
% \newtheorem{ax}[thm]{Axioma}

%en ingles..
 \newtheorem{thm}{Theorem}[section]
 \newtheorem{lem}[thm]{Lemma}
 \newtheorem{prop}[thm]{Proposition}
 \newtheorem*{prop*}{Proposition}
 \newtheorem{cor}[thm]{Corollary}
 \newtheorem{conj}[thm]{Claim}
 \newtheorem{ej}[thm]{Example}
 \newtheorem{ax}[thm]{Axiom}


% % Renombrar Comandos
% \newcommand{\bd}[1]{\mathbf{#1}}
% \newcommand{\R}{\mathbb{R}}      
% \newcommand{\Z}{\mathbb{Z}}     
% \newcommand{\N}{\mathbb{N}}
% \newcommand{\Ra}{\rightarrow}
% \newcommand{\U}{\mathcal{U}}
% \newcommand{\col}[1]{\left[\begin{matrix} #1 \end{matrix} \right]}
% \newcommand{\comb}[2]{\binom{#1^2 + #2^2}{#1+#2}}

% %Highlight text
% \usepackage{color}
% \newcommand{\hilight}[1]{\colorbox{yellow}{#1}}

%\documentclass[12pt]{amsart}
%\usepackage{geometry} % see geometry.pdf on how to lay out the page. There's lots.
%\geometry{a4paper} % or letter or a5paper or ... etc
%% \geometry{landscape} % rotated page geometry
%
%% See the ``Article customise'' template for come common customisations

\title{Notes model theory}
\author{OZ}
%\date{} % delete this line to display the current date

%%% BEGIN DOCUMENT
\begin{document}

\maketitle
%\tableofcontents
%
%\section{Ultraproducts}
%\subsection{}
\subsubsection*{Intro}
Ehrenfeucht-Fra\"iss\'e games characterise the expressive power of logical languages \cite{ams}. Every Ehrenfeucht-Fra\"iss\'e game is an ultraproduct \cite{models}, a back-and-forth method for showing isomorphism between countably infinite structures, but only defined for finite structures in finite model theory. There are highly elaborated algorithmic techniques in AI for model-checking properties on generic objects like finite structures of some specified vocabulary (e.g. \cite{Carrillo}), each of which somewhat builds on the relationship between game theoretic methods for showing complexity results, such as EF games, and its logical structure and interpretations. Thereof a deeper relationship can be described using ultrafilters and ultralimits. 


We want a way to study discrete objects with continuous methods. This can be achieved via an ultraproduct construction. 

Suppose $\alpha$ is a non-principal ultrafilter on $I\in 2^{\mathbb{N}}$ (actually, on $2^I\subseteq 2^\mathbb{N}$ to be more precise) and $A_i$, $B_i$ are structures of a countable vocabulary $\tau$. An ultrafilter is a filter that is maximal w.r.t. the partial ordering of any Boolean algebra on which the filter has been defined, in particular w.r.t. the partial order on the lattice $\mathbf{2} := \langle 2^{\mathbb{N}}, \subseteq\rangle$ (which is clearly a Boolean algebra), when thinking of ultrafilters as subsets of the natural numbers $\mathbb{N}$ with the finite intersection property (i.e. $\inf\{\alpha\subseteq\mathbb{N}: |\alpha|<\aleph_0\} \neq 0$). An ultrafilter $\alpha$ is non-principal if it contains no finite set. 
 
 
 Assuming that $\alpha$ is a filter containing an infinite descending sequence with empty intersection, we have the following result.
\begin{thm*}
If $A_i \equiv B_i$ for all $i\in I$, then player $B$ has a winning strategy in $EF_{\omega} (\prod_{i} A_i/\alpha, \prod_{i} B_i/\alpha)$.
\end{thm*}

\subsubsection*{Ultrafilters}
Given a nonempty set $L$ define a binary relation $\succ$ on it by forcing the following formulas true:
\begin{itemize} 
\item[1.] Reflexivity: $\forall x (x\succ x)$
\item[2.] Transitivity: $\forall xyz (x\succ y)\land (y\succ z)\rightarrow (x\succ z)$
\item[3.] Antisymmetry: $\forall xy (x\succ y)\land (y\succ x)\rightarrow (x=y)$ 
\end{itemize}
The pair $\langle L, \succ \rangle$ is called a \emph{poset} and $\succ$ a \emph{partial order} on $L$. For every two-element subset $\{x,y\}$ of $L$ we define its \emph{meet} and \emph{join} as $x\sqcap y := \inf_{\succ}\{x,y\}$ and $x\sqcup y := \sup_{\succ}\{x,y\}$, respectively. A \emph{lattice} is a poset where every two-element subset has meet and join (i.e. $\forall\{x,y\}\in 2^L(\exists (x\sqcap y)\in L\land\exists (x\sqcup y)\in L)$ is satisfied).

A \emph{filter} is a subset $F$ of a lattice $L$ which contains all the \emph{successors} (if we name ``$\succ$" the ``successor" relation) of any member of $F$ (i.e. $\forall xy(y\in F)\land(x\succ y)\rightarrow(x\in F)$ holds). An \emph{ultrafilter} $\alpha$ is a maximal filter with respect to the usual partial order relation that one can always define in any Boolean algebra (particularly in $2^L$). Ultrafilters are characterised by the next equivalence.

\begin{lem*}
If $\alpha$ is a filter in a Boolean algebra $B$, $\alpha$ is an ultrafilter if and only if for each $x\in B$ either $x\in\alpha$ or $1-x\in \alpha$, but not both.
\end{lem*}

\subsubsection*{Ultraproducts}
An ultraproduct is a mathematical construction that permits us to take the limit of any discrete object in any discrete space and actually build a limit object which exists in the ultraproduct of the spaces which is for instance a limit space. The ultraproduct construction is a universal way to go from discrete to continuous, back and forth, carrying along the axioms and the operations.

Now, \cite{Hirsch}

\subsubsection*{$\defaultL$o\'s's lemma}
\begin{thm*}[$\defaultL$o\'s's Lemma]
If $\alpha$ is an ultrafilter and $\varphi$ a first-order formula, then the ultraproduct of models of $\varphi$ indexed by any index set $I\in \alpha$ is a model of $\varphi$, i.e. \[\prod_i A_i / \alpha\models\varphi\Leftrightarrow\{i\in I: A_i\models\varphi\}\in \alpha.\]
\end{thm*}


%\nocite{*}
\bibliographystyle{plain}
\bibliography{ultraproduct}
\end{document}